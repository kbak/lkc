\documentclass{chi2009}
\usepackage{times}
\usepackage{url}
\usepackage{graphics}
\usepackage{color}
\usepackage[pdftex]{hyperref}
\usepackage{graphicx}
\usepackage{tabularx}
\usepackage{booktabs}

\newcommand{\docTitle}{Improving Usability of the Linux Kernel Configuration Tools}
\newcommand{\docKeywords}{usability, linux, kernel, configuration}

\hypersetup{%
pdftitle={\docTitle},
pdfauthor={Kacper Bak},
pdfkeywords={\docKeywords},
bookmarksnumbered,
pdfstartview={FitH},
colorlinks,
citecolor=black,
filecolor=black,
linkcolor=black,
urlcolor=black,
breaklinks=true,
}
\newcommand{\comment}[1]{}
\definecolor{Orange}{rgb}{1,0.5,0}
\newcommand{\todo}[1]{\textsf{\textbf{\textcolor{Orange}{[[#1]]}}}}

\pagenumbering{arabic}  % Arabic page numbers for submission.  Remove this line to eliminate page numbers for the camera ready copy

\begin{document}
% to make various LaTeX processors do the right thing with page size
\special{papersize=8.5in,11in}
\setlength{\paperheight}{11in}
\setlength{\paperwidth}{8.5in}
\setlength{\pdfpageheight}{\paperheight}
\setlength{\pdfpagewidth}{\paperwidth}

% use this command to override the default ACM copyright statement 
% (e.g. for preprints). Remove for camera ready copy.
\toappear{Submitted for CS 889 - Open Source Usability.}

\title{\docTitle}
\numberofauthors{2}
\author{
  \alignauthor Kacper Bak\\
    \affaddr{Generative Software Development Lab}\\
    \affaddr{University of Waterloo, Canada}\\
    \email{kbak@gsd.uwaterloo.ca}
  \alignauthor Karim Ali\\
    \affaddr{PLG Group}\\
    \affaddr{University of Waterloo, Canada}\\
    \email{karim@uwaterloo.ca}
}

\maketitle

\begin{abstract}
Tailoring a Linux kernel to one's needs has been one of the most cumbersome tasks a GNU/Linux user can do. There have been many attempts to overcome this problem by introducing smarter configuration tools. Those tools, however, still lack some important features, which discourages users from using them. In this project, we plan to address the problem of usability of the Linux kernel configuration tools. Our aim is to identify the major usability issues with current tools, propose a better user interface and evaluate it on a group of Linux enthusiasts.
\end{abstract}

\keywords{\docKeywords} 

\category{H.5.2}{Information Interfaces and Presentation}{Miscellaneous}%[Optional sub-category]

\section{Introduction}

\section{Problem Domain}\label{sec:problem}

% motivation: static configuration (customizing kernel), dynamic configuration

\section{Linux Kernel Configuration Tool}\label{sec:lkc}

% rationale for the design

% xconfig vs lkc

% scalability

% implementation

\section{Evaluation}\label{sec:evaluation}

% participants characterization

\subsection{Threats to Validity}

\paragraph{External Validity}

\paragraph{Internal Validity}

\section{Future Work}\label{sec:futurework}

% \section{Future Work} - we'll place this section in the final report
% \todo{what do you think about creating a graphical tool that goes with you step by step and configures a kernel? basically you answer some questions, press next, and finally click install to install the new kernel. It might be appreciated by beginners. Is it too simplistic? would it be useful? it improves usability, but are there many users who need such a limited tool?}

% backends

\section{Related Work}\label{sec:relatedwork}
% literature review

\section{Conclusion}\label{sec:conclusion}

\nocite{*}

\bibliographystyle{abbrv}
\bibliography{doc}

\end{document}
